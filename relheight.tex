\documentclass[10pt]{article}
%\usepackage[utf8]{inputenc}
\usepackage{amsmath}    % need for subequations
\usepackage{amsfonts}
\usepackage{amssymb}  % needed for mathbb  OK
\usepackage{bigints}
\usepackage{graphicx}   % need for figures
\usepackage{subfig}
\usepackage{verbatim}   % useful for program listings
\usepackage{color}      % use if color is used in text
%\usepackage{subfigure}  % use for side-by-side figures
\usepackage{parskip}
\usepackage{float}
\usepackage{courier}
\usepackage{exercise}
\usepackage{sistyle}
\SIthousandsep{,}
%\usepackage{numprint}
\setlength\parindent{0pt}

\newtheorem{prop}{Proposition}


\renewcommand{\DifficultyMarker}{}
\newcommand{\AtBeginExerciseHeader}{\hspace{-21pt}}  %-0.2pt
\renewcommand{\ExerciseHeader}{\AtBeginExerciseHeader\textbf{\ExerciseName~\ExerciseHeaderNB} \ExerciseTitle}
\renewcommand{\AnswerHeader}{\large\textbf{\AnswerName~\ExerciseHeaderNB}\smallskip\newline}
\setlength\AnswerSkipBefore{1em}

\usepackage{makeidx}
\makeindex
\usepackage[nottoc]{tocbibind}


\usepackage[colorlinks = true,
          linktocpage=true,
            pagebackref=true, % add back references to bibliography
            linkcolor = red,
            urlcolor  = blue,
            citecolor = red,
 %           refcolor  =red,
            anchorcolor = blue]{hyperref}
\definecolor{dkgreen}{rgb}{0,0.6,0}
\definecolor{gray}{rgb}{0.5,0.5,0.5}
\definecolor{mauve}{rgb}{0.58,0,0.82}
\definecolor{index}{rgb}{0.88,0.32,0}

%------- source code settings
\usepackage{listings}
\lstset{frame=tb,
  language=Python,
  aboveskip=3mm,
  belowskip=3mm,
  showstringspaces=false,
  columns=flexible,
  basicstyle={\small\ttfamily},
  numbers=none,
  numberstyle=\tiny\color{gray},
  keywordstyle=\color{blue},
  commentstyle=\color{dkgreen},
  stringstyle=\color{mauve},
  breaklines=true,
  breakatwhitespace=true,
  tabsize=3
}

%-----------------------------------------------------------------

\usepackage{blindtext}
\usepackage{geometry}
 \geometry{
 a4paper,
 total={170mm,257mm},
 left=20mm,
 top=20mm,
 }


\setlength{\baselineskip}{0.0pt} 
\setlength{\parskip}{3pt plus 2pt}
\setlength{\parindent}{20pt}
\setlength{\marginparsep}{0.0cm}
\setlength{\marginparwidth}{0.0cm}
\setlength{\marginparpush}{0.0cm}
\setlength{\tabcolsep}{4pt}
\renewcommand{\arraystretch}{1.4} %%%
\newtheorem{theorem}{Theorem}[section]
\newtheorem{lemma}[theorem]{Lemma}
\newtheorem{proposition}[theorem]{Proposition}
\newtheorem{corollary}[theorem]{Corollary}

\newenvironment{proof}[1][Proof]{\begin{trivlist}
\item[\hskip \labelsep {\bfseries #1}]}{\end{trivlist}}
\newenvironment{definition}[1][Definition]{\begin{trivlist}
\item[\hskip \labelsep {\bfseries #1}]}{\end{trivlist}}
\newenvironment{example}[1][Example]{\begin{trivlist}
\item[\hskip \labelsep {\bfseries #1}]}{\end{trivlist}}
\newenvironment{remark}[1][Remark]{\begin{trivlist}
\item[\hskip \labelsep {\bfseries #1}]}{\end{trivlist}}

\newcommand{\qed}{\nobreak \ifvmode \relax \else
      \ifdim\lastskip<1.5em \hskip-\lastskip
      \hskip1.5em plus0em minus0.5em \fi \nobreak
      \vrule height0.75em width0.5em depth0.25em\fi}


\begin{document}

\hypersetup{linkcolor=blue}
%inserting a glossary entry in gloss: \gls{gls:keyword1} \\


\begin{center}
{\Large \bf{How the Brain Estimates Distances, Heights and Velocities,\\
\vspace{1.0ex} with Computer Vision Applications} 
%\quad \\ \addvspace{1ex} 
%with Spectacular Applications
}  \\
% \addvspace{1ex}
%Stochastic Processes and Simulations -- Volume 1
\addvspace{5ex}
\end{center}
 
\begin{center}
Vincent Granville, Ph.D. \\
 vincentg@MLTechniques.com\\
 \href{https://mltechniques.com/}{www.MLTechniques.com}\\
\quad 
Version 1.0, December 2024  \\ \quad \vspace{4ex}\\
\end{center}

\hypersetup{linkcolor=red}

A moving plane or car far away from you seems to move slowly and appears to accelerate as it gets closer to you, even though its
speed is constant. Yet you are able to tell the approximate speed at all times, whether the vehicle is 100 yards or a mile away. Likewise, when looking at a skyscraper, at ground level, 100 feet away from the building, the windows in the upper floors seem smaller than those at the bottom, even though they are of the same size. 

While I don't know exactly how the brain makes the corrections, in my case it seems that it uses a rather small mapping table between perceived and actual distances, using interpolation methods to increase accuracy. The purpose of this article is to explain the elegant mathematical formula used to generate that table. In the case of the human brain (mine at least), the table is built over time via reinforcement learning, and could be empty at birth: 
information retrieval to get the correct distance is just a memory call enhanced with simple interpolation, but not involving advanced mathematics. 

\section{Apparent versus real distance}

Due to perspective, the perceived distance to an object is $1/d$, where $d$ is the real distance. Let's now look 
at a the height $H$ of a skyscraper, seen from a distance $d$ at ground level.  Based on Figure~\ref{fig:gmlp9utheight}, we have $H = d\tan \theta$. 

%-----------------------------vince/riemann2and3.mp4
\begin{figure}[H]
\centering
\includegraphics[width=0.73\textwidth]{height2.png}
\caption{Height $H$ of segment BC measured from A at ground level}
\label{fig:gmlp9utheight}
\end{figure}
%imgpy9979_2and3.PNG
%-------------------------

\noindent The actual height $H$ is a large sum of tiny increments $\Delta H$, each proportional to the constant $d$. To measure $H$, think of it as being in an elevator located at $A$, moving at constant speed, and counting the milliseconds until the horizontal line of length $d$ between you and the building, does not hit the building anymore when your elevation is greater than $H$. Now, let's turn to polar coordinates.

For the perceived height $h_r$, you are stuck at location $A$ (the origin), looking up at various angles $\theta$ from bottom to top. Each angle $\theta$ contributes a tiny height $\Delta h_r(\theta)$ to the total perceived height. Using the weight $w(\theta) = \lambda / d_\theta$ where $\lambda > 0$ is a constant, and the notation
$q(\theta) = d/\cos^2\theta$, we can write $H$ and $h_r$ as two versions of the same integral: the former with a fixed weight, and the latter with a variable weight, as follows:
$$
H  = \int_0^\theta q(\theta) d\theta = d\tan\theta, \quad 
h_r  = \int_0^\theta w(\theta)q(\theta) d\theta.
$$
The choice of $w(\theta)$ is dictated by the fact that the perceived distances are inversely  
proportional to the actual ones (the $d_\theta$'s) with a distinct $d_\theta$ for each $\theta$. 
Also, $q(\theta)=1/\cos^2\theta$ is the only continuous function that integrates to $H=d\tan\theta$ for all
$0\leq\theta<\pi/2$; you cannot choose another one.
Finally, since $d_\theta = d/\cos\theta$, we also have $w(\theta)q(\theta) = \lambda/\cos \theta$. Thus, the perceived
 height at angle $\theta$ is 

\begin{equation}
h_r(\theta) = \lambda \int_0^\theta \frac{1}{\cos \theta} d\theta = \lambda
\log\Big[\tan\Big(\frac{\pi}{4}+\frac{\theta}{2}\Big)\Big], \quad\quad 0\leq \theta < \frac{\pi}{2}, \label{fw3kvfjy}
\end{equation}
where $\lambda > 0$ is a constant not depending on $\theta$. The minimum value is $h_r(0) = 0$, while the maximum
 is infinite and reached when $\theta = \pi/2$.  Since $H=d\tan\theta$, the perceived height $g_r(H)$ -- defined as a function of the real height $H$ -- is equal to 
\begin{equation}
g_r(H)=h_r(\theta) = h_r\Big(\arctan\frac{H}{d}\Big) = 
\lambda
\log\Big[\tan\Big(\frac{\pi}{4}+\frac{1}{2}\arctan\frac{H}{d}\Big)\Big]. \label{oxo9v}
\end{equation}
With simple trigonometry, it is easy to simplify Formula~(\ref{oxo9v}) to obtain~(\ref{ps2fc}),
 where $\sinh^{-1}$ is the inverse hyperbolic sine function. From there, by inverting~(\ref{ps2fc}),
 one can express the actual height $H$ as a function of the perceived height $g_r(H)$, see Formula~(\ref{ps2fccx}).
Here, $g_r$ is defined as a function of $H$ while $h_r$ is a function of $\theta$. 

\begin{align}
%g_r(H) & = \lambda\log\Bigg(\frac{H + \sqrt{H^2+d^2}}{d}\Bigg),  \label{ps2fc} \\
g_r(H) & = \lambda\cdot\sinh^{-1}\Big(\frac{H}{d}\Big),  \label{ps2fc} \vspace{1ex}\\
H & = d\cdot \sinh\Big(\frac{g_r(H)}{\lambda}\Big). \label{ps2fccx}
%H & = \frac{\exp(g_r(H)/\lambda)-\exp(-g_r(H)/\lambda)}{2}. \label{ps2fccx}
%g_r(H)   & =  \frac{d+H +\sqrt{d^2 + H^2}}{d-H+\sqrt{d^2 + H^2}}, \label{ps2fc}  \\
\end{align}
Now is it easy to answer the following question. For a skyscraper seen from a distance $d$, what is the required height $H'$ for the building to appear $\mu$ times taller, given that its real height is $H$? 
Let $\rho_\mu(H) = H'/H$ be~the multiplier,
 also denoted as  $\rho_\mu(H, d)$. 
It does not depend on $\lambda$. To obtain $H'$, you need to solve the equation
$g_r(H')=\mu g_r(H)$ where $H'$ is the unknown. Using~(\ref{ps2fc}), the solution is
\begin{equation}
H' = d\cdot \sinh\Big(\mu \cdot \sinh^{-1}\Big(\frac{H}{d}\Big)\Big). \label{turton}
\end{equation}
I computed some values of the multiplier $\rho_\mu(H)$ for various $\mu$ and $H$, assuming
 $d=1$. See Table~\ref{table:kurdisre}. Obviously, $\rho_\mu(H) = 1$ when $\mu=1$, 
and $\rho_\mu(H)\approx\mu$ when $H$ is small compared to $d$.
When point A is close to B in Figure~\ref{fig:gmlp9utheight} and $H$ is large, that is when
$\theta$ is close to $\pi/2$, then $\rho_\mu(H)$ grows very fast as
 $\mu$ increases. All this is confirmed by personal visual experiences.
\vspace{1ex}

\begin{table}[H]
\centering
\renewcommand{\arraystretch}{1.0}
\begin{tabular}{ccc|ccc|ccc|ccc} 
 \hline
 $\mu$ & $H$ & $\rho_\mu(H)$ & $\mu$ & $H$ & $\rho_\mu(H)$ &$\mu$ & $H$ & $\rho_\mu(H)$ &$\mu$ & $H$ & $\rho_\mu(H)$ \\[0.5ex] 
 \hline 
\hline 
   &  &    &  &    &  &    &  &  \\  [-2ex]
0.50&0.50&0.49&0.50&1.00&0.46&0.50&2.00&0.39&0.50&3.00&0.35\\
1.00&0.50&1.00&1.00&1.00&1.00&1.00&2.00&1.00&1.00&3.00&1.00\\
1.50&0.50&1.57&1.50&1.00&1.74&1.50&2.00&2.15&1.50&3.00&2.54\\
2.00&0.50&2.24&2.00&1.00&2.83&2.00&2.00&4.47&2.00&3.00&6.32\\
2.50&0.50&3.03&2.50&1.00&4.47&2.50&2.00&9.23&2.50&3.00&15.71\\
3.00&0.50&4.00&3.00&1.00&7.00&3.00&2.00&19.00&3.00&3.00&39.00\\

 \hline
\end{tabular}
\caption{Values of $\rho_\mu(H)$ depending on $\mu, H$, with $d=1$}
\label{table:kurdisre}
\end{table}
\noindent
Table~\ref{table:kurdisre} should be interpreted as follows. Assume that you are at ground level, 100 feet away from a 300 feet~tall building ($H=3$). The builders keep adding new floors and when you visit again, the building appears to be 1.5 taller
($\mu = 1.50$) seen from the same vantage point. In fact, this is an illusion due to perspective: in reality, it is actually $2.54$ taller since $\rho_{1.50}(3) = 2.54$. The real height is now 
$762 = 2.54 \times 300$ feet. It is interesting to note that if $d=1$ and $\mu$ is an odd integer,
then $\rho_{\mu}(H)$ is an integer when $H$ is an integer. The easy proof relies on~(\ref{turton}) and the fact that 
$\rho_{\mu}(H, d) = H'/H$, leading to the following result:
\begin{equation}
\rho_{\mu}(H,d) = \frac{d}{2H} \Bigg[
\Bigg(\frac{\sqrt{H^2 + d^2} +H}{d}\Bigg)^{\mu} 
- \Bigg(\frac{\sqrt{H^2 + d^2} - H}{d}\Bigg)^{\mu}
\Bigg]. \label{rotores}
\end{equation}
Formula~(\ref{rotores}) is a direct consequence of the definitions of the hyperbolic sine function and its inverse, combined with simple manipulations. From there, we obtain the linear recurrence relation
\begin{equation}
\rho_{\mu+1}(H, d) =\frac{1}{d^2}\Big[ (2d^2+4H^2) \rho_{\mu-1}(H, d)  +   (d^2-2H^2)  \rho_{\mu-3}(H, d)  \Big] ,
\end{equation}
with initial conditions 
$$
\rho_{0}(H, d)  = 0, \quad
\rho_{1}(H, d)  = 1, \quad
\rho_{2}(H,d)  = \frac{1}{d}\sqrt{4H^2+4d^2}, \quad
\rho_{3}(H,d)  = \frac{1}{d^2}(4H^2+3d^2). 
$$


\section{Computer vision problems}\label{f8v2kde}

The article {\em How can we Explain Perspective Calculations Simply?}, available
 \href{https://www.metabunk.org/threads/how-can-we-explain-perspective-calculations-simply.10570/}{here}, 
provides an introduction. See also {\em The Size of an Object at a Distance} posted 
\href{https://www.1790.us/-search-rescuehumanhead.html}{here} by the
Department of Homeland Security and intended for coast guards. The Wikipedia entry about angular
diameter, \href{https://en.wikipedia.org/wiki/Angular_diameter}{here}, shows applications in astronomy.  
 
If you rotate the triangle in Figure~\ref{fig:gmlp9utheight} by 90$^\circ$ to the right, then the discussion 
remains the same with height replaced by length, as in Figure~\ref{fig:gmlvt2kft}. Despite
 appearances in the picture, railroad ties have the same width and are equally spaced.  So, the distortions
 due to perspective and corrected by our brain are not illusions but also captured by any vision
 system. They can be accurately measured. 
In Figure~\ref{fig:gmlvt2kft}, we do not have the context to determine the exact lengths, but the methodology presented here allows you to obtain correct ratios. Likewise, Formula~(\ref{fw3kvfjy}) has a scale parameter $\lambda$ that cannot 
 be estimated based on the picture alone, yet relative lengths can be computed exactly.
\vspace{1ex}

%-----------------------------vince/riemann2and3.mp4
\begin{figure}[H]
\centering
\includegraphics[width=0.73\textwidth]{rail2b.png}
\caption{Railroad ties are equally spaced despite appearances}
\label{fig:gmlvt2kft}
\end{figure}
%imgpy9979_2and3.PNG
%-------------------------

To measure velocities, get the location of a moving locomotive at two different times, compute the
 apparent distance $g_r(H)$ between the two locations on the picture, then get the real distance $H$
 with Formula~(\ref{ps2fccx}).  In this case, $d$ is the distance between your vantage point above ground, and the railroad; $\lambda$  is the scale parameter. You can use the same method to measure accelerations, and
 get it implemented in real-time. 

I conclude with the following observation. If you double the angle $\theta$ from 30$^\circ$ to 60$^\circ$ in 
Figure~\ref{fig:gmlp9utheight}, then~the actual height $H$ is multiplied by 3, regardless of $d$. However the perceived height
 is multiplied by only 2.3975, regardless of $\lambda$. The exact value is $\log(2+\sqrt{3})/\log\sqrt{3}$, computed according to Formula~(\ref{fw3kvfjy}).
Here $\sqrt{3}=\tan(60^\circ)$ and $2+\sqrt{3}=\tan(75^\circ)$.
 
%xxx use AI to answer

%---




%\bibliographystyle{plain} % We choose the "plain" reference style
%\bibliography{refstats} % Entries are in the refs.bib file in same directory as the tex file
%\printindex
%\pagebreak







\hypersetup{linkcolor=red} % red %
\hypersetup{linkcolor=red}



\end{document}