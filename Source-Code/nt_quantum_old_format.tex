\documentclass[10pt]{article}
%\usepackage[utf8]{inputenc}
\usepackage{amsmath}    % need for subequations
\usepackage{amsfonts}
\usepackage{amssymb}  % needed for mathbb  OK
\usepackage{bigints}
\usepackage{graphicx}   % need for figures
\usepackage{subfig}
\usepackage{verbatim}   % useful for program listings
\usepackage{color}      % use if color is used in text
%\usepackage{subfigure}  % use for side-by-side figures
\usepackage{parskip}
\usepackage{float}
\usepackage{courier}
\usepackage{exercise}
\usepackage{sistyle}
\SIthousandsep{,}
%\usepackage{numprint}
\setlength\parindent{0pt}

\newtheorem{prop}{Proposition}


\renewcommand{\DifficultyMarker}{}
\newcommand{\AtBeginExerciseHeader}{\hspace{-21pt}}  %-0.2pt
\renewcommand{\ExerciseHeader}{\AtBeginExerciseHeader\textbf{\ExerciseName~\ExerciseHeaderNB} \ExerciseTitle}
\renewcommand{\AnswerHeader}{\large\textbf{\AnswerName~\ExerciseHeaderNB}\smallskip\newline}
\setlength\AnswerSkipBefore{1em}

\usepackage{makeidx}
\makeindex
\usepackage[nottoc]{tocbibind}


\usepackage[colorlinks = true,
          linktocpage=true,
            pagebackref=true, % add back references to bibliography
            linkcolor = red,
            urlcolor  = blue,
            citecolor = red,
 %           refcolor  =red,
            anchorcolor = blue]{hyperref}
\definecolor{dkgreen}{rgb}{0,0.6,0}
\definecolor{gray}{rgb}{0.5,0.5,0.5}
\definecolor{mauve}{rgb}{0.58,0,0.82}
\definecolor{index}{rgb}{0.88,0.32,0}

%------- source code settings
\usepackage{listings}
\lstset{frame=tb,
  language=Python,
  aboveskip=3mm,
  belowskip=3mm,
  showstringspaces=false,
  columns=flexible,
  basicstyle={\small\ttfamily},
  numbers=none,
  numberstyle=\tiny\color{gray},
  keywordstyle=\color{blue},
  commentstyle=\color{dkgreen},
  stringstyle=\color{mauve},
  breaklines=true,
  breakatwhitespace=true,
  tabsize=3
}

%-----------------------------------------------------------------

\usepackage{blindtext}
\usepackage{geometry}
 \geometry{
 a4paper,
 total={170mm,257mm},
 left=20mm,
 top=20mm,
 }


\setlength{\baselineskip}{0.0pt} 
\setlength{\parskip}{3pt plus 2pt}
\setlength{\parindent}{20pt}
\setlength{\marginparsep}{0.0cm}
\setlength{\marginparwidth}{0.0cm}
\setlength{\marginparpush}{0.0cm}
\setlength{\tabcolsep}{4pt}
\renewcommand{\arraystretch}{1.4} %%%
\newtheorem{theorem}{Theorem}[section]
\newtheorem{lemma}[theorem]{Lemma}
\newtheorem{proposition}[theorem]{Proposition}
\newtheorem{corollary}[theorem]{Corollary}

\newenvironment{proof}[1][Proof]{\begin{trivlist}
\item[\hskip \labelsep {\bfseries #1}]}{\end{trivlist}}
\newenvironment{definition}[1][Definition]{\begin{trivlist}
\item[\hskip \labelsep {\bfseries #1}]}{\end{trivlist}}
\newenvironment{example}[1][Example]{\begin{trivlist}
\item[\hskip \labelsep {\bfseries #1}]}{\end{trivlist}}
\newenvironment{remark}[1][Remark]{\begin{trivlist}
\item[\hskip \labelsep {\bfseries #1}]}{\end{trivlist}}

\newcommand{\qed}{\nobreak \ifvmode \relax \else
      \ifdim\lastskip<1.5em \hskip-\lastskip
      \hskip1.5em plus0em minus0.5em \fi \nobreak
      \vrule height0.75em width0.5em depth0.25em\fi}


\begin{document}

\hypersetup{linkcolor=blue}
%inserting a glossary entry in gloss: \gls{gls:keyword1} \\


\begin{center}
{\Large \bf{%Number Theory Breakthrough Sparks New AI, Quantum, \vspace{1ex}\\ HPC, and Cryptography Research
LLM Challenge with Petabytes of Data to Prove\vspace{1ex}\\ Famous Number Theory Conjecture} 
%\quad \\ \addvspace{1ex} 
%with Spectacular Applications
}  \\
% \addvspace{1ex}
%Stochastic Processes and Simulations -- Volume 1
\addvspace{5ex}
\end{center}
 
\begin{center}
Vincent Granville, Ph.D. \\
 vincentg@MLTechniques.com\\
 \href{https://mltechniques.com/}{www.MLTechniques.com}\\
\quad 
Version 1.0, March 2025  \\ \quad \vspace{4ex}\\
\end{center}

\hypersetup{linkcolor=red}

%\tableofcontents

%\vspace{6ex}

%\section{Introduction}

\begin{abstract}

In my recent article ``Piercing the Deepest Mathematical Mystery", I paved the way to proving a famous multi-century old conjecture: are the digits of major mathematical constant such as $\pi, e, \log 2, \sqrt{2}$ evenly distributed?
No one before ever managed to prove even the most basic trivialities, such as whether the proportion of `0' or `1' exists in the binary expansions of any of these constants, or if it oscillates indefinitely between 0\% and 100\%. 
Here I provide an overview of the new framework built to uncover deep results about the digit distribution of Euler's number $e$, discuss the latest developments, share an upgraded version of the code, and feature new potential research areas across multiple fields, arising from my discovery. 
\end{abstract}

\section{Introduction}

My first article on the topic, published in February 2025, is posted  \href{https://mltblog.com/3zsnQ2g}{here} as paper 51.
The methodology relies~on material discussed  in my book ``Gentle Introduction to
Chaotic Dynamical Systems"~\cite{vgchaos}. The original article is now part of that book.  
I used bit strings to represent binary digit sequences, establishing the link to large language models (LLMs). Here, I will stick to integers instead, to simplify the presentation.
The main idea is to work with iterated \textcolor{index}{self-convolutions}\index{self-convolution}. For fixed integers $n$ and $x$, iteratively  define the sequence $\{ S(n, k, x) \}$ with the recursion
\begin{equation}
S(n, k+1, x) = S^2(n, k, x), \quad k=0,1,2\dots  \label{gfrna9u}
\end{equation}
with  $S(n, 0, x) = 2^n + x$. The initial value $S(n, 0, x)$ is called the \textcolor{index}{seed}\index{seed}. 
We are interested in $x=\pm1$ or $x=3$. It follows that the first $\tau(n)$ binary digits of $S(n, n, x)$ match those of $\exp(x)$, with $\tau(n) = n +O(1)$. This remains true if at each iteration $k$, we \textcolor{index}{truncate}\index{truncation} $S(n, k, x)$ and only keep the first $2n$ binary digits on the left, so that $S(n,k,x)$ is an integer with $2n$ digits at all times.

A more traditional approach consists in multiplying the seed $S(n, 0, x)$ by an integer power of 2, positive or negative, so that it lies in the interval $[1, 2[$. Then, replace Formula~(\ref{gfrna9u}) by the recursion

\begin{equation}
S(n, k+1, x) = \begin{cases}
\displaystyle  
S^2(n,k,x) & \text{if $1\leq S(n, k, x) < \sqrt{2}$},\\
\frac{1}{2} S^2(n, k, x)  & \text{if $\sqrt{2}\leq S(n, k, x) <2$}. \label{beurre4}
\end{cases}
\end{equation}

\noindent Then, the sequence $\{S(n,k,x) \}$ indexed by $k$, with fixed $n$ and $x$, is an \textcolor{index}{ergodic}  quadratic \textcolor{index}{dynamical system}. It is \textcolor{index}{homeomorphic}\index{homeomorphic} both to the \textcolor{index}{logistic map}\index{logistic map}  and the \textcolor{index}{dyadic map}, with $S(n,k,x)\in [1,2[$ at all times. In short, it is
 a mapping from $[1, 2[$ onto itself. Its \textcolor{index}{invariant measure}\index{invariant measure} is the 
\textcolor{index}{reciprocal distribution}, a probability distribution defined as $F(z) = \log_2 z$ for $1\leq z < 2$.
The theory, including the connection to \textcolor{index}{normal numbers}\index{normal number},  is discussed in chapter 6 in~\cite{vgchaos}. 

Whether using (\ref{gfrna9u}) or (\ref{beurre4}), the end results are the same, as we are only interested in the 
 first $n$ binary digits of $S(n, k, x)$ on the left side, for $k=0, 1, 2$ and so on. I now introduce the 
\textcolor{index}{digit sum}\index{digit sum function} function,
 also known as \textcolor{index}{digit count}\index{digit count} for binary digits, as $\zeta   _S(n, k, x)$. For a fixed $n$ and $x$, it counts the number of `1' in the first $n$ binary digits of $S(n, k, x)$. The normalized version
\begin{equation}
\zeta   _S^*(n,k, x) = \frac{1}{n} \zeta   _S(n, k, x) \label{oxyver}
\end{equation}
takes a value between 0 and 1. It represents the proportion of `1' in the first $n$ digits of $S(n,k,x)$. 
Therefore, $\zeta   _S^*(n, n, x)$ represents the proportion of `1' in the first $\tau(n)$ digits in the binary expansion of $\exp(x)$. It is~not difficult to prove that
 $\tau(n) = n + O(1)$. Finally, to obtain theoretical bounds on the proportion of `1' in $\exp(x)$, here with
$x=\pm 1$ or $x= 3$, you need to study the spectacular behavior of $\zeta   _S^*(n, n, x)$ for fixed $n$ and $x$, and then
let $n\rightarrow\infty$. The remaining of the article focuses on this problem, and its various applications
 to~quantum dynamics, cryptography, AI and LLM evaluation in particular, dynamical systems, high performance computing, and so on.  

\section{Spectacular behavior of the digit sum function}

Before digging into this problem, it is worth noticing that my approach is radically different from all previous attempts to
prove that the binary digits of numbers such as $e$ or $\pi$ are evenly distributed. They all failed even~at proving much weaker results. The main innovations in my framework are:
\vspace{1ex}
\begin{itemize}
\item Focusing on very few digits on the left side, rather than many digits on the right side and ignoring the leftmost part of the digit expansion. Yet, as $n\rightarrow\infty$, `very few' eventually becomes infinite.  
\item To uncover patterns, experimenting with numbers as large as $2^n+1$ at power $2^n$, with $n=10^5$, far beyond the capability of any computer system. This is possible thanks to the truncation mechanism.
\item Working with systems exhibiting strong and fascinating patterns that can be precisely quantified, rather than navigating in the dark. It helps visualize the steps necessary to build a solid proof. 
\item Building on expertise with discrete dynamical systems, thus having a clear idea of what to expect, both in terms of leverage and challenges.  
\item Working with the most basic seeds leading to a unique behavior. We do not care if the seed 
 is attached to the main invariant measure or not, as we stop at iteration $k=n$ in $S(n, k, x)$, that is, at the very end 
of the non-chaotic regime of the underlying dynamical system, just before full chaos sets in. 
\end{itemize}
\vspace{1ex}



\subsection{Potential scenarios}

Moving forward, I use $n=10^5$ in all my illustrations. Unless otherwise specified, the integers $n$ and $x$ are fixed. As with~all similar dynamical systems (dyadic map, in particular), the main function of interest, in this case the digit~sum $\zeta   _S(n, k, x)$ as a function of $k$, can have a wide variety of shapes depending on the seed. Here, the seed corresponds to $S(n,k,x)$ with $k=0$. That is, $S(n, 0, x) = 2^n + x$. 

%-----------------------------vince/riemann2and3.mp4
\begin{figure}[H]
\centering
\includegraphics[width=0.77\textwidth]{nte1.png}
\caption{Digit sum $\zeta_S(n,k,x)$ with $x=1$, $n=10^5$, and $k$ on the X-axis}
\label{fig:ggk3vb5dlk}
\end{figure}
%imgpy9979_2and3.PNG
%-------------------------

Figure~\ref{fig:ggk3vb5dlk} features an extremely rare behavior, and one of the easiest to predict. Nevertheless, it is  the only one of interest to prove deep results about the binary digits of $e$. The seed corresponds to a string with $n-1$ consecutive `0's, with a `1' at both ends. I will use this case as a basis to describe alternate, much more common scenarios.   

At $k = 3n/4$, the \textcolor{index}{bifurcation phase}\index{bifurcation phase} starts and lasts until
 $k=n$. And when $k\geq n$, we are in the full \textcolor{index}{chaotic phase}\index{chaotic phase}, with the 
 proportion of `1' oscillating around 50\%. Interestingly, at $k=n$, the width of the horizontal band visible in the top right corner
 is conjectured to be $O(\sqrt{n\log\log n})$ at most, a consequence of the \textcolor{index}{law of the iterated logarithm}\index{law of the iterated logarithm}, and $O(\sqrt{n})$ on average, a consequence of the 
\textcolor{index}{central-limit theorem}\index{central-limit theorem}.
 
 The limit of the digit sum function when $n\rightarrow\infty$, when plotted on a fixed-size window, 
is a continuous time process with infinitesimal increments. Yet, unlike a Brownian motion (the limit of
 a random walk), it is nowhere continuous, jumping from one state to another as $k$ increases. 
The finite number of potential states also increases as $k$ increases. Each state has its own color in Figure~\ref{fig:ggk3vb5dlk}, 
 and corresponds to a particular residue of $k$ in a specific \textcolor{index}{congruential class}\index{congruential class}. 
Despite being nowhere continuous, the graph appears to be much less erratic than a Brownian motion. In particular,
 the segments are connected with no apparent discontinuity, a property specific to the seed $2^n + x$ with $x=\pm 1$ or $x=3$, but not true in other cases. Due to the potential states, the multi-branch function is called a 
\textcolor{index}{quantum function}\index{quantum function}.
\vspace{1ex}

%-----------------------------vince/riemann2and3.mp4
\begin{figure}[H]
\centering
\includegraphics[width=0.77\textwidth]{nte1_zoom.png}
\caption{Zoom in on the right part of Figure~\ref{fig:ggk3vb5dlk}}
\label{fig:g609jb5dlk_green}
\end{figure}
%imgpy9979_2and3.PNG
%-------------------------

I can now share a number of possible scenarios for the behavior of the digit sum function, depending on the seed, for a large variety of seeds. The list below does not cover all the cases, but only those that are either the most common, or related to our discussion. The seed must have at least $n$ bits, starting with `1', and ending with `1' unless its length is infinite. 
\vspace{1ex}
\begin{itemize}
\item The most common case by far is when the seed is a random string with $n$ bits. Then there is no warm-up phase. Instead, we enter the chaotic phase at the very beginning, when $k=0$, instead of  $k=n$ in Figure~\ref{fig:ggk3vb5dlk}.
\item If the seed  consists mostly of `0', with the proportion of `1' above (say) 10\%, then the chaotic phase is reached in very few iterations. This is the most interesting case for cryptography.
\item Seeds with $n=10^5$ bits with only 3 to 5 bits set to `1's at random locations, are useful for research. 
The chaotic phase can be reached pretty fast, but sometimes with gaps in the warm-up phase. See Figure~\ref{fig:g609jb5dlk}. 
\item If the seed is a real number $2^{q/p}$ with $p, q$ coprime integers, $p$ an odd prime, then there is no chaotic phase. 
The digit sum function is periodic. The length of the period is  the \textcolor{index}{multiplicative order}\index{multiplicative order} of 2 modulo $p$.
\end{itemize}
\vspace{1ex}

\noindent The seeds $S(n,0,x)$ that we are working with to establish deep results about the digits of $e$, are among the very few that require so many iterations ($k=n$) before reaching the chaotic phase. 
       
\subsection{Dynamics of the unbroken bifurcation process}

I conjecture that the seeds $S(n,0,x) = 2^n+x$ with $x=\pm 1$ or $x=3$, results in no gap during the non-chaotic phase, unlike Figure~\ref{fig:g609jb5dlk}. If gaps were present, proving the desired result would 
potentially be more difficult. I~use the word ``unbroken bifurcation process" to describe the absence of gap.

If you look at Figure~\ref{fig:ggk3vb5dlk}, new forks and changes in the slope occur at specific locations $k$ on the X-axis. The vertical dashed lines show these locations. They are identical regardless of $n$, and whether $x=\pm 1$ or $x=3$. These locations correspond to $k/n\approx \rho_1, \, \rho_2, \, \rho_3$ and so on, with
\begin{equation}
\rho_1 = \frac{1}{2},  \quad \rho_2 = \frac{2}{3}, \quad \rho_3 = \frac{3}{4}, 
\quad \rho_4 = \frac{4}{5}, \quad \rho_5 = \frac{5}{6}, \quad
\rho_6 = \frac{6}{7}, \quad \dots \label{pu5deroi}
\end{equation}
These approximated values are extremely accurate. On the right on the X-axis, beyond $k/n =\rho_7$, the number of change points start exploding with more and more segments, shorter and shorter, eventually evolving in a very narrow range just before reaching the chaotic phase with about 50\% of `1'. Within each vertical band delimited by consecutive vertical dashed lines, the number of segments, from left to right,
 is respectively equal to 1, 1, 2, 2, 4, 12 and so on.  
Except for the first two values, these are factorials multiplied by 2. Since $S(n, k, x) = 2^n + x$ at power $2^k$, as $k$ increases this combinatorial explosion is not surprising.
Then, 
based on~(\ref{pu5deroi}), the widths of each band, from left to right,
 are proportional to 
$$\frac{1}{1\cdot 2}, \quad \frac{1}{2\cdot 3}, \quad \frac{1}{3\cdot 4},\quad
\frac{1}{4\cdot 5}, \quad \dots
$$ 
 The proportionality factor is $n$.  Paths are determined by congruential residues. For instance, the green path in Figure~\ref{fig:g609jb5dlk_green}
 corresponds to values of $k$ such that $k \equiv 4 \bmod{12}$.
Finally, the shape of the graph, for a specific $x$, stays the same regardless of $n$. But different values of $x$ produce different shapes. 
\vspace{1ex}
%Figure~\ref{fig:g609jb5dlk_green} shows a more granular view, zooming in on Figure~\ref{xxxxxxxxx} 

\section{Case studies}

Figures~\ref{fig:ggk3vb5dlk} and~\ref{fig:g609jb5dlk_green} feature the case $x=1$. Here, I show the dynamics of the digit sum function for the cases
 $x=-1$ and $x=3$, associated respectively with the binary digits of $e^{-1}$ and $e^3$. Then 
Figure~\ref{fig:g609jb5dlk} shows the results when using a seed consisting of a string of length $n=10^5$, all `0' except  three `1' at random locations, plus a 
 `1' at both ends. Finally, I discuss what happens when averaging values within each vertical band delimited by vertical dashed lines 
in Figures~\ref{fig:ggk3vb5dlk} and~\ref{fig:g609jb5dlk_green}.
\vspace{1ex}

%-----------------------------vince/riemann2and3.mp4
\begin{figure}[H]
\centering
\includegraphics[width=0.78\textwidth]{nte-1.png}
\caption{Digit sum $\zeta_S(n,k,x)$ with $x=-1$, $n=10^5$, and $k$ on the X-axis}
\label{fig:g609jb5dlkem1}
\end{figure}
%imgpy9979_2and3.PNG
%-------------------------

The case $x=-1$ is featured in Figures~\ref{fig:g609jb5dlkem1} and~\ref{fig:g609jb5dlkem1cx}. The seed $2^n-1$ consists of $n$ bits all `1', without any `0'. The graph is more elaborate than the cases $x=1$ and $x=3$, starting at $k=0$ with all `1', with the proportion going down 
to 50\% as we reach $k = n/2$, but then bouncing back up before eventually stabilizing around 50\% when $k>n$. Despite the different behavior compared to $x=1$ or $x=3$, the vertical dashed lines indicating new bifurcations and slope changes, still have the same abscissas. 

At first glance, it seems more difficult to prove a formal theorem about the digit distribution for this case. However, due to the various segments being more spread out, possibly with fewer hidden features, it certainly makes the patterns easier to quantify, maybe facilitating a proof for the digits of $e^{-1}$, compared to
 $e$ or $e^3$.

%-----------------------------vince/riemann2and3.mp4
\begin{figure}[H]
\centering
\includegraphics[width=0.77\textwidth]{nte-1_zoom.png}
\caption{Zoom in on the right part of Figure~\ref{fig:g609jb5dlkem1}}
\label{fig:g609jb5dlkem1cx}
\end{figure}
%imgpy9979_2and3.PNG
%-------------------------


%-----------------------------vince/riemann2and3.mp4
\begin{figure}[H]
\centering
\includegraphics[width=0.78\textwidth]{nte3.png}
\caption{Digit sum $\zeta_S(n,k,x)$ with $x=3$, $n=10^5$, and $k$ on the X-axis}
\label{fig:g609jb5dlke3}
\end{figure}
%imgpy9979_2and3.PNG
%-------------------------

Figures~\ref{fig:g609jb5dlke3} and~\ref{fig:g609jb5dlke3z} correspond to the case $x=3$. The seed has $n+1$ bits, all `0' except the first one on the left  and the two rightmost equal to `1'. Despite the seed being apparently more complex than the case $x=1$, having three `1' rather than two, the bifurcation phase (before chaos) evolves in a much narrower range, mostly with increases in the proportion of `1' over time until about 50\% is reached,  with very few, moderate setbacks. It could make a proof easier
 than for the case $x=1$. However many features may be hidden, making pattern analysis more complicated. 
Again, the positions of the vertical dashed lines are unchanged.


%-----------------------------vince/riemann2and3.mp4
\begin{figure}[H]
\centering
\includegraphics[width=0.78\textwidth]{nte3_zoom.png}
\caption{Zoom in on the right part of Figure~\ref{fig:g609jb5dlke3}}
\label{fig:g609jb5dlke3z}
\end{figure}
%imgpy9979_2and3.PNG
%-------------------------


%-----------------------------vince/riemann2and3.mp4
\begin{figure}[H]
\centering
\includegraphics[width=0.77\textwidth]{ntrnd_zoom.png}
\caption{Digit sum: seed with $n=10^5$ bits with five `1' at random locations}
\label{fig:g609jb5dlk}
\end{figure}
%imgpy9979_2and3.PNG
%-------------------------

Figure~\ref{fig:g609jb5dlk} features a seed not related to the digits of any known number. It is not part of a scheme where increasing $n$ results
 in convergence of $S(n,n,x)$ to some constant. Unlike the previous examples, the locations of the `1's in the seed of length $n=10^5$ are spread randomly, yet with a `1' at each end. The total number of `1' is five. Thus, depending on the locations of the random `1's, the integer $x$ can be a very large number, possibly much larger than $2^{n/2}$ yet much smaller than $2^n$. Nevertheless, this case is very interesting because it shows that gaps sometimes occur. While mild in most cases, in this example they are rather substantial. Note that the X-axis is truncated to magnify the gaps, as they appear early in the process, with the rightmost visible one  at $k < n/6$.

%-----------------------------vince/riemann2and3.mp4
\begin{figure}[H]
\centering
\includegraphics[width=0.77\textwidth]{nte1b.png}
\caption{Averaging the segments within each vertical band in Figure~\ref{fig:ggk3vb5dlk}}
\label{fig:g609jb5dlk1b}
\end{figure}
%imgpy9979_2and3.PNG
%-------------------------
%--

Finally, if you average the segments withing each vertical band delimited by consecutive vertical dashed lines in Figure~\ref{fig:ggk3vb5dlk}, you obtain Figure~\ref{fig:g609jb5dlk1b}. Thus, we are approaching a straight line starting at $k=4n/5$, reaching the chaotic phase at $k=n$ for the number of `1's
 in the first $n$ binary digits of $2^n+1$ at power $2^k$. Thus, with about 50\% of `1' when $n=k$, for the first $n$ digits of $e$. 

To produce Figure~\ref{fig:g609jb5dlk1b}, I actually used a much simpler technique to get an approximation: the curve is a moving average based on a window with 12 consecutive values of $k$. In addition, all the pictures and results featured here were produced using the Python code
 in section~\ref{pyuferdma}.  
 My algorithm is  based on formula~(\ref{gfrna9u}) combined with the truncation mechanism. 
Also, I double-check the correctness in the digits of $e$ with an external library. 



%-------------------------------------

\section{Applications and AI Challenge with petabytes dataset}

In this section, I first provide a quick overview of potential applications, 
with recent references. Then I discuss an interesting challenge: using
\textcolor{index}{large language models}\index{large language model} (LLMs) to leverage my research and uncover deeper insights, test and compare their mathematical and pattern detection capabilities, based on the digit dataset and via reasoning. After all, the iterates
 $S(n, k, x)$ are highly correlated strings similar to sentences as in English prose or DNA sequences, 
here with an alphabet consisting of two letters: `0' and `1'. Let's start with the references: 
\vspace{1ex}

\begin{itemize}
\item The framework presented here relies on discrete \textcolor{index}{quadratic dynamical systems}\index{quadratic dynamical systems}. 
This family also includes the \textcolor{index}{logistic map}\index{logistic map} and the example
 discussed in~\cite{reapoh}.   For additional references, see my book on chaos and dynamical systems~\cite{vgchaos}.

\item Showing that the binary digits are evenly distributed is the first step towards proving that $e$ is a \textcolor{index}{normal number}\index{normal number}. Andrew Granville and Davig Bailey~\cite{nt1} are good references~on this topic. 
  For recent publications on normal numbers, see Verónica Becher~\cite{vero}
and~\cite{dffs}. One of best results know for any major math constant is the fact that the proportion of ones in the first $n$ binary digits of $\sqrt{2}$ is larger than $\sqrt{2n}$, 
see~\cite{sqrt2newd}. 

\item The digit sum or digit count functions (both are identical for binary digits) is also known as the \textcolor{index}{Hamming weight}\index{Hamming weight}, with a fast algorithm described \href{https://stackoverflow.com/questions/14555607/number-of-bits-set-in-a-number}{here} and a full chapter in~\cite{hw}.  
The Wolfram entry for the \textcolor{index}{digit sum}\index{digit sum function} (see \href{https://mathworld.wolfram.com/DigitCount.html}{here}) features an exact closed-form formula for the number of digits equal to 1 in the binary expansion of any integer, with more references. 
For a discussion on the \textcolor{index}{carry digit}\index{carry digit function} function (a 
\textcolor{index}{2-cocycle}\index{cocycle}) that propagates 1's from right to left in the successive
 iterations $S(n, k, x)$, see~\cite{dcfnt9, alan}. 

\item An interesting application of the digit sum  
 is featured in~\cite{vmo} in the context of genotype maps,
 with processes not unlike the dynamical systems discussed in this article, and \textcolor{index}{blancmange curves}\index{blancmange curve}
 almost identical to Figure 3.3 in my book on numeration systems~\cite{vgchaos}.  

\item There is a connection to \textcolor{index}{quantum maps}\index{quantum map} and 
\textcolor{index}{quantum cryptography}\index{quantum cryptography} \cite{qc19,qc24}.
For \textcolor{index}{PRNGs}\index{PRNG} (pseudo-random generators) based on irrational numbers, 
see chapter 13  in~\cite{vgxllm2} or chapter 4 in~\cite{vgchaos}.
Finally, if you use an arbitrary seed instead of $S(n, 0, x) = 2^n+1$, 
you obtain strings that look random, after very few iterations.

\item \textcolor{index}{Deep neural networks}\index{deep neural network} have been used  to identify the underlying model of dynamical systems, based on available data produced by 
simulations or from real life observations, see~\cite{botuliee4j, bkw3r,retuloi}. In our case, the model would be a simple formula that generates the values
 of the digit sum function, to study its asymptotic properties. 
\end{itemize}
\vspace{1ex}

\subsection{AI challenge}

\noindent The last item in the bullet list brings an interesting challenge. The idea is to use AI and large language models to get the full picture about the patterns. The goal is to formally prove the deepest possible results that you can get from my framework, about the binary digits of the 
 number $e$, or related numbers such as $e^{-1}$ or $e^3$. In particular, we want to fully understand and accurately quantify the bifurcation phase shown in all the figures. Questions to answer include
\vspace{1ex}
\begin{itemize}
\item The number of segments (referred to as \textcolor{index}{quantum states}\index{quantum state}) in each vertical band delimited by successive vertical dashed lines, for instance in Figures~\ref{fig:g609jb5dlk_green}, \ref{fig:g609jb5dlkem1cx}, and \ref{fig:g609jb5dlke3z}.
We want to find a general formula for the number of segments based on the location on the X-axis. 
\item Which values of $k$ correspond to each segment. We know that the answer depends on the residues of $k$ modulo specific integers linked to factorials. What are these residues and the modulo classes in question? It seems like and exact, simple, and general answer can be obtained.
The number of segments within a vertical bands is directly linked to the modulo classes in question. 
\item The slopes of these segments, and the mean slope when averaged within a vertical band, with a general formula applicable to any location on the X-axis. Show that when the slope is moving in the wrong direction (away from the 50\% target in the proportion of `1'), it can only do so for so long. 
\item Confirm the absence of gaps when $x=\pm 1$ or $x=3$, by contrast to Figure~\ref{fig:g609jb5dlk}. What gap-free cases share in common? 
Also confirm and extend Formula~(\ref{pu5deroi}) to cover the entire range from $k=0$ to $k=n$.
Finally, what is the exact shape of the envelope pictured in Figure~\ref{fig:g60fll8vgp}, showing the minimum and maximum potential values 
 for the digit sum function at any location $k$, with $0\leq k\leq n$. 

\end{itemize}
\vspace{1ex}

%-----------------------------vince/riemann2and3.mp4
\begin{figure}[H]
\centering
\includegraphics[width=0.77\textwidth]{nte1_envelop.png}
\caption{Upper/lower bounds for $\zeta_S(n,k,x)$ with $x=1$, $n=10^5$, and $k$ on the X-axis}
\label{fig:g60fll8vgp}
\end{figure}
%imgpy9979_2and3.PNG
%-------------------------

\noindent Rather than smart guesses or predictions, we want actual proofs whenever possible. The purpose of using AI is not to get approximations to model parameters, but exact values when possible, and even a generic formula that generates all the
 values, such as Formula~(\ref{pu5deroi}). AI may also be able to identify other applications underlined by the same dynamical models,
 providing valuable information. This approach is discussed in a recent paper~on data-driven model discovery  \cite{ddmd24}.

\subsection{The dataset}

Large language models (LLMs) are trained to predict the next tokens given previous tokens, based on a large database of text. In our case, for a fixed $n$ and $x$, each $S(n, k, x)$ is a string consisting of $2n$ bits. In LLM parlance, it is a block of tokens, each token consisting of (say) 
 512 bits. We want to predict $S(n, k+1, x), S(n, k+2, x)$ and so on, based on $S(n, k, x), S(n, k-1, x)$ and so on. The training set
 consists of all $S(n, k, x)$ with $0\leq k \leq n$. We only care about the first $n$ bits on the left. Indeed are only interested in the
 digit sum function computed on these first $n$ bits, at any $k$. The problem is trivial until $k$ gets close to $n$. 
The closer to $n$, the harder.
Thus we can restrict training and predictions to  $k /n> 0.85$. The predictions must all be exact in this case, and the methodology 
 must use cross-validation. 

Detecting the rules that make correct predictions is even more important. Or identifying other well-studied dynamical systems that have a very similar behavior, with known properties. In short, anything that can lead to formally proving a deep result about the digit sum, is highly valuable. 
Better, an actual proof if some LLMs can come up with one. Perhaps something like this: the proportion of `1' in the binary digits of $e$ is above 10\%. This in itself would be a phenomenal result, beating everything obtained previously, by a very long shot.  Proving that it is exactly 50\% is the best that we could expect.

For that purpose, one can create a dataset consisting of $S(n, k, x)$ strings with $n=10^7$, for 100 different values of $x$, some causing gaps and some not in the digit sum function. This requires generating $2 \times 10^7 \times 10^7 \times 10^2$ bits, that is, 2.5 petabytes. LLMs that
 can detect the patterns with a much smaller dataset, and those that~still perform well very close to $k=n$, should be rewarded. 
The code in section~\ref{pyuferdma}, with $n=10^5$, is a starting point to create the dataset. It runs fast (about a minute) on a small laptop. 
Each value of $x$ can be run in parallel. 
I uploaded the values of the digit sum function for $x=1$, $n=10^5$, and $0\leq k \leq n$, on GitHub, 
\href{https://raw.githubusercontent.com/VincentGranville/Experimental-Math-Number-Theory/refs/heads/main/Source-Code/digit_sum_e1.txt}{here}. 

%https://raw.githubusercontent.com/VincentGranville/Experimental-Math-Number-Theory/refs/heads/main/Source-Code/digit_sum_e1.txt
% openAI, grok, perplexity, deepseek

\section{Python code}\label{pyuferdma}

The Python code \texttt{number\_theory\_fast\_v2.py} is also on GitHub, 
\href{https://github.com/VincentGranville/Experimental-Math-Number-Theory/blob/main/Source-Code/number_theory_fast_v2.py}{here}. 
In lines \textcolor{gray}{121--122}, I use the Mpmath library to compute
 the digits of $e$, to double check that my algorithm yields the correct digits as advertised.
The value of $x$ is determined in lines \textcolor{gray}{74--80}, with the variable \texttt{p} playing the role of $x$ unless it is set to 0. 
When set to 0, it generates a random seed. The variable \texttt{H} determines the upper limit for $k$, that is, the number
 of iterations. Finally, the envelope in Figure~\ref{fig:g60fll8vgp} was produced using a plot (line~\textcolor{gray}{155} in the code),
 rather than a scatterplot (line~\textcolor{gray}{154} in the code).
% https://github.com/VincentGranville/Experimental-Math-Number-Theory/blob/main/Source-Code/number_theory.py
\vspace{1ex}

\begin{lstlisting}[numbers=left]
n = 100000 
H = int(1.15*n)  

def assign_color(k):

    if k%12 == 0:
        color = 'lime'
    elif k%12 == 2:
        color = 'white' 
    elif k%12 == 4:
        color = 'black'
    elif k%12 == 6:
        color = 'cyan'
    elif k%12 == 8:
        color = 'gold'
    elif k%12 == 10:
        color = 'orange'
    elif k%12 == 1:
        color = 'magenta'
    elif k%12 == 3:
        color = 'paleturquoise'
    elif k%12 == 5: 
        color = 'khaki'
    elif k%12 == 7:
        color = 'yellow'
    elif k%12 == 9:
        color = 'mistyrose'
    elif k%12 == 11:
        color = 'orangered'
    return(color)


#--- 1. Main

import gmpy2
import numpy as np

kmin = 0.00 * n  # do not compute digit count if k <= kmin
kmax = 1.15 * n  # do not compute digit count if k >= kmax
kmax = min(H, kmax)

# precision set to L bits to keep at least about n correct bits till k=kmax
ctx = gmpy2.get_context()  
L = n + int(kmax+1)
ctx.precision = L     

# p = +1: leading to e (Euler's number), about n correct bits after n iter
#         seed with n+1 bits, all 0 except rightmost and leftmost
# p = +3: leading to cube of e, about n correct bits after n iter
#         seed with n+1 bits, all 0 except 2 rightmost and leftmost
# p = -1: leading to inverse of e, about n correct bits after n iter
#         seed with n bits, all 1
# p must be an integer != 0; use p=0 for random seed

p = 1   # try -1, 0 (random seed), 1, 3 

def create_random_seed(n, cnt1):

    # create random seed of length n-1 with cnt1 '1' at random locations
    # and add a 1' at both ends; cnt1 must be <= n-1
 
    cnt1 = min(cnt1, n-1)
    numpy_seed = 453    # numpy seed to initiate numpy PRNG, not the model seed
    np.random.seed(numpy_seed)
    random_locations = np.random.choice(np.arange(1, n),size=cnt1,replace=False)
    prod = 2**n + 1     # seed with n+1 '0' except a '1' at both ends
    for position in random_locations:
        # add the cnt1 random '1's between both ends
        prod += 2**int(position)
    return(prod)


# create seed with n+1 bits if p>=0, or n bits if p<0
if p != 0:
    prod = gmpy2.mpz(2**n + p)           
else:
    # random seed with number of '1' in seed to cnt1+2
    # test: set n = 30000; cnt1 = 0, 3, 4, 5, 100  and see what happens!
    cnt1 = 3  
    prod = create_random_seed(n, cnt1)  

# local variables
arr_count1 = []
arr_colors = []
xvalues = []
ecnt1 = -1

OUT = open("digit_sum.txt", "w")

for k in range(1, H+1): 

    prod = prod*prod
    pstri = bin(int(prod))
    stri = pstri[0: L+2]  
    prod = int(stri, 2)
    prod = gmpy2.mpz(prod) 

    if k > kmin and k < kmax:   
        stri = stri[2:]
        if k == n:
            e_approx = stri
        estri = stri[0:n]   # leftmost n digits
        ecnt1 = estri.count('1')
        arr_count1.append(ecnt1)
        color = assign_color(k)
        arr_colors.append(color)
        xvalues.append(k)
        OUT.write(str(k) + "\t" + str(ecnt1) + "\n")
 
    if k%1000 == 0:
        print("%3d %3d" %(k, ecnt1))

OUT.close()


#--- 2. Compute bits of e and count correct bits in my computation

from mpmath import mp

# Set precision to L binary digits
mp.dps = int(L*np.log2(10))
e_value = (mp.e)**abs(p)  # Get e^|p| in decimal

if p > 0:
    # Convert e_value to binary string
    e_binary = bin(int(e_value * (2 ** n)))[2:] 

elif p < 0: 
    e_iapprox = int(e_approx, 2)  # convert string e_approx to integer
    e_ivalue = int(2**(2*n) * e_value)
    one = e_iapprox * e_ivalue
    e_approx = bin(one)[2:]  
    e_binary = "1" * (2*n)  # string of 2n bits, all '1'

if p != 0:
    k = 0
    while e_approx[k] == e_binary[k]:
        k += 1
    # e_binary should be equal to e_approx up to about n bits  
    print("\n%d correct digits (n = %d)" %(k, n))


#--- 3. Create the main plot

import matplotlib.pyplot as plt
import matplotlib as mpl
import numpy as np

mpl.rcParams['axes.linewidth'] = 0.5
plt.rcParams['xtick.labelsize'] = 8
plt.rcParams['ytick.labelsize'] = 8
plt.rcParams['axes.facecolor'] = 'black'

plt.scatter(xvalues, arr_count1,s=0.01, c=arr_colors) 
# plt.plot(xvalues, arr_count1,linewidth=0.04, c='gold') 

plt.axhline(y=n/2,color='red',linestyle='--',linewidth=0.6,dashes=(5,10))   
plt.axhline(y=n/5, color='black', linestyle='--', linewidth = 0.6, dashes=(5, 10))
plt.axvline(x=n, color='red', linestyle='-', linewidth = 0.6, dashes=(5, 10))

for k in range(1,15):
    plt.axvline(x=k*n/(k+1),c='gray',linestyle='--',linewidth=0.6,dashes=(5, 10))

if p > 0:
    # we start with about 0% of 1 going up to about 50%
    ymax = 0.52 * n
    plt.ylim([-0.01 * n, ymax]) 
elif p < 0:
    # we start with 100% of 1 going down to about 50%
    ymax = 1.01 * n
    plt.ylim([0.40 * n, ymax])
elif p == 0: 
    ymax = 1.00 * n
    plt.ylim([-0.01 * n, ymax])
plt.xlim([kmin, kmax])

plt.show()

#--- 4. Create the average plot

arr_avg = []
arr_xval = []
arr_count1 = np.array(arr_count1)

for k in range(0, int(kmax-12)):
    y_avg = np.average(arr_count1[k:k+12])
    arr_avg.append(y_avg)
    arr_xval.append(k)

plt.scatter(arr_xval,arr_avg,s=0.0002, c='gold')
plt.axhline(y=n/2,color='red',linestyle='--',linewidth=0.6,dashes=(5,10))   
plt.axhline(y=n/5, color='black', linestyle='--', linewidth = 0.6, dashes=(5, 10))
plt.axvline(x=n, color='red', linestyle='-', linewidth = 0.6, dashes=(5, 10))
plt.xlim(0.00*n, kmax-12)
plt.ylim(-0.01*n, ymax)

for k in range(1,15):
    plt.axvline(x=k*n/(k+1),c='gray',linestyle='--',linewidth=0.6,dashes=(5, 10))

plt.show()
\end{lstlisting}


\bibliographystyle{plain} % We choose the "plain" reference style
\bibliography{refstats} % Entries are in the refs.bib file in same directory as the tex file
%\printindex
%\pagebreak







\hypersetup{linkcolor=red} % red %
\hypersetup{linkcolor=red}



\end{document}